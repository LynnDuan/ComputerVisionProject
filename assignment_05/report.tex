% !TeX spellcheck = en_US
%%%%%%%%%%%%%%%%%%%%%%%%%%%%%%%%%%%%%%%%%
% Lachaise Assignment
% LaTeX Template
% Version 1.0 (26/6/2018)
%
% This template originates from:
% http://www.LaTeXTemplates.com
%
% Authors:
% Marion Lachaise & François Févotte
% Vel (vel@LaTeXTemplates.com)
%
% License:
% CC BY-NC-SA 3.0 (http://creativecommons.org/licenses/by-nc-sa/3.0/)
% 
%%%%%%%%%%%%%%%%%%%%%%%%%%%%%%%%%%%%%%%%%

%----------------------------------------------------------------------------------------
%	PACKAGES AND OTHER DOCUMENT CONFIGURATIONS
%----------------------------------------------------------------------------------------

\documentclass{article}
\usepackage{bm}

%%%%%%%%%%%%%%%%%%%%%%%%%%%%%%%%%%%%%%%%%
% Lachaise Assignment
% Structure Specification File
% Version 1.0 (26/6/2018)
%
% This template originates from:
% http://www.LaTeXTemplates.com
%
% Authors:
% Marion Lachaise & François Févotte
% Vel (vel@LaTeXTemplates.com)
%
% License:
% CC BY-NC-SA 3.0 (http://creativecommons.org/licenses/by-nc-sa/3.0/)
% 
%%%%%%%%%%%%%%%%%%%%%%%%%%%%%%%%%%%%%%%%%

%----------------------------------------------------------------------------------------
%	PACKAGES AND OTHER DOCUMENT CONFIGURATIONS
%----------------------------------------------------------------------------------------

\usepackage{amsmath,amsfonts,stmaryrd,amssymb} % Math packages

\usepackage{enumerate} % Custom item numbers for enumerations

\usepackage[ruled]{algorithm2e} % Algorithms

\usepackage[framemethod=tikz]{mdframed} % Allows defining custom boxed/framed environments

\usepackage{listings} % File listings, with syntax highlighting
\lstset{
	basicstyle=\ttfamily, % Typeset listings in monospace font
}

%----------------------------------------------------------------------------------------
%	DOCUMENT MARGINS
%----------------------------------------------------------------------------------------

\usepackage{geometry} % Required for adjusting page dimensions and margins

\geometry{
	paper=a4paper, % Paper size, change to letterpaper for US letter size
	top=2.5cm, % Top margin
	bottom=3cm, % Bottom margin
	left=2.5cm, % Left margin
	right=2.5cm, % Right margin
	headheight=14pt, % Header height
	footskip=1.5cm, % Space from the bottom margin to the baseline of the footer
	headsep=1.2cm, % Space from the top margin to the baseline of the header
	%showframe, % Uncomment to show how the type block is set on the page
}

%----------------------------------------------------------------------------------------
%	FONTS
%----------------------------------------------------------------------------------------

\usepackage[utf8]{inputenc} % Required for inputting international characters
\usepackage[T1]{fontenc} % Output font encoding for international characters

\usepackage{XCharter} % Use the XCharter fonts

%----------------------------------------------------------------------------------------
%	COMMAND LINE ENVIRONMENT
%----------------------------------------------------------------------------------------

% Usage:
% \begin{commandline}
%	\begin{verbatim}
%		$ ls
%		
%		Applications	Desktop	...
%	\end{verbatim}
% \end{commandline}

\mdfdefinestyle{commandline}{
	leftmargin=10pt,
	rightmargin=10pt,
	innerleftmargin=15pt,
	middlelinecolor=black!50!white,
	middlelinewidth=2pt,
	frametitlerule=false,
	backgroundcolor=black!5!white,
	frametitle={Command Line},
	frametitlefont={\normalfont\sffamily\color{white}\hspace{-1em}},
	frametitlebackgroundcolor=black!50!white,
	nobreak,
}

% Define a custom environment for command-line snapshots
\newenvironment{commandline}{
	\medskip
	\begin{mdframed}[style=commandline]
}{
	\end{mdframed}
	\medskip
}

%----------------------------------------------------------------------------------------
%	FILE CONTENTS ENVIRONMENT
%----------------------------------------------------------------------------------------

% Usage:
% \begin{file}[optional filename, defaults to "File"]
%	File contents, for example, with a listings environment
% \end{file}

\mdfdefinestyle{file}{
	innertopmargin=1.6\baselineskip,
	innerbottommargin=0.8\baselineskip,
	topline=false, bottomline=false,
	leftline=false, rightline=false,
	leftmargin=2cm,
	rightmargin=2cm,
	singleextra={%
		\draw[fill=black!10!white](P)++(0,-1.2em)rectangle(P-|O);
		\node[anchor=north west]
		at(P-|O){\ttfamily\mdfilename};
		%
		\def\l{3em}
		\draw(O-|P)++(-\l,0)--++(\l,\l)--(P)--(P-|O)--(O)--cycle;
		\draw(O-|P)++(-\l,0)--++(0,\l)--++(\l,0);
	},
	nobreak,
}

% Define a custom environment for file contents
\newenvironment{file}[1][File]{ % Set the default filename to "File"
	\medskip
	\newcommand{\mdfilename}{#1}
	\begin{mdframed}[style=file]
}{
	\end{mdframed}
	\medskip
}

%----------------------------------------------------------------------------------------
%	NUMBERED QUESTIONS ENVIRONMENT
%----------------------------------------------------------------------------------------

% Usage:
% \begin{question}[optional title]
%	Question contents
% \end{question}

\mdfdefinestyle{question}{
	innertopmargin=1.2\baselineskip,
	innerbottommargin=0.8\baselineskip,
	roundcorner=5pt,
	nobreak,
	singleextra={%
		\draw(P-|O)node[xshift=1em,anchor=west,fill=white,draw,rounded corners=5pt]{%
		Question \theQuestion\questionTitle};
	},
}

\newcounter{Question} % Stores the current question number that gets iterated with each new question

% Define a custom environment for numbered questions
\newenvironment{question}[1][\unskip]{
	\bigskip
	\stepcounter{Question}
	\newcommand{\questionTitle}{~#1}
	\begin{mdframed}[style=question]
}{
	\end{mdframed}
	\medskip
}

%----------------------------------------------------------------------------------------
%	WARNING TEXT ENVIRONMENT
%----------------------------------------------------------------------------------------

% Usage:
% \begin{warn}[optional title, defaults to "Warning:"]
%	Contents
% \end{warn}

\mdfdefinestyle{warning}{
	topline=false, bottomline=false,
	leftline=false, rightline=false,
	nobreak,
	singleextra={%
		\draw(P-|O)++(-0.5em,0)node(tmp1){};
		\draw(P-|O)++(0.5em,0)node(tmp2){};
		\fill[black,rotate around={45:(P-|O)}](tmp1)rectangle(tmp2);
		\node at(P-|O){\color{white}\scriptsize\bf !};
		\draw[very thick](P-|O)++(0,-1em)--(O);%--(O-|P);
	}
}

% Define a custom environment for warning text
\newenvironment{warn}[1][Warning:]{ % Set the default warning to "Warning:"
	\medskip
	\begin{mdframed}[style=warning]
		\noindent{\textbf{#1}}
}{
	\end{mdframed}
}

%----------------------------------------------------------------------------------------
%	INFORMATION ENVIRONMENT
%----------------------------------------------------------------------------------------

% Usage:
% \begin{info}[optional title, defaults to "Info:"]
% 	contents
% 	\end{info}

\mdfdefinestyle{info}{%
	topline=false, bottomline=false,
	leftline=false, rightline=false,
	nobreak,
	singleextra={%
		\fill[black](P-|O)circle[radius=0.4em];
		\node at(P-|O){\color{white}\scriptsize\bf i};
		\draw[very thick](P-|O)++(0,-0.8em)--(O);%--(O-|P);
	}
}

% Define a custom environment for information
\newenvironment{info}[1][Info:]{ % Set the default title to "Info:"
	\medskip
	\begin{mdframed}[style=info]
		\noindent{\textbf{#1}}
}{
	\end{mdframed}
}
 % Include the file specifying the document structure and custom commands

%----------------------------------------------------------------------------------------
%	ASSIGNMENT INFORMATION
%----------------------------------------------------------------------------------------

\title{CMPT726: Assignment \#1} % Title of the assignment

\author{Lin Duan 301369502 \& Jingwen Zhang 301368992 \\ \texttt{duanlind@sfu.ca \& jingwen\_zhang\_2sfu.ca }} % Author name and email address

\date{Simon Fraser University--- \today} % University, school and/or department name(s) and a date

%----------------------------------------------------------------------------------------

\begin{document}
	
	\maketitle % Print the title
	
	%----------------------------------------------------------------------------------------
	%	INTRODUCTION
	%----------------------------------------------------------------------------------------
	
	\section{Implementation} 
	\subsection{checkinliner} 
	For the point $x$ in the previous images , and the relevant point $x^{'}$ in the next image,  if $x^{T} F x^{'} \leq  d$, then return true.
	\subsection{Findfundmental}
	Firstly, normalized the previous subset and the next subset. Then use the normalized subsets to form a matrix $A$, which $$A = \begin{pmatrix}
	x_{1}^{'}x_{1} & x_{1}^{'}y_{1} & x_{1}^{'} & y_{1}^{'}x_{1} & y_{1}^{'}y_{1} & y_{1}^{'} & x_1 & y_1 & 1 \\
	\vdots & \vdots & \vdots & \vdots & \vdots & \vdots & \vdots & \vdots  \\
	x_{8}^{'}x_{8} & x_{8}^{'}y_{8} & x_{8}^{'} & y_{8}^{'}x_{8} & y_{8}^{'}y_{8} & y_{8}^{'} & x_8 & y_8 & 1
	\end{pmatrix}$$
	And then use SVD to decompose the $A$ to get the last column of $V$, and resize it into 3*3 matrix. Then use SVD to decompose this matrix to get $U_1, W_1, V_{1}^{T}$and set the $W_1(3,3)$ into 0.0. Then compute the fundamental matrix with $F= U_1 W_1 V_{1}^{T}$. And de-normalize the fundamental matrix with $F = N^{T}FN$
	
	\subsection{Findessential}
	$E = K^{T} F K$
	\subsection{relativepose}
	This function is supposed to get four relative pose between two frames. Firstly, use SVD to decompose the essential matrix $E$ to get $U, W, V^{T}$, and set $W_{tmp}$ with $W_{tmp}(0,1) = -1, W_{tmp}(1,0) = 1, W_{tmp}(2,2) = 1$. 
	
	Secondly, we could get two candidate rotation matrix with $R_1 = UW_{tmp}V^T$ and $R_1 = UW_{tmp}^{T}V^T$. And check their determinant, if $|R| < 0$ then $R = -R$. And the candidate transition matrix$T_1$ and $T_2$ should be the last column of $U$ and its negative one. 
	
	Finally, the candidate relative pose should be $[R_1, T_1]$ or $[R_2, T_1]$ or $[R_1, T_2]$ or $[R_2, T_2]$
	\subsection{triangulation}
	This function is supposed to get the correct relative pose between two frames. First, we set $(x,y)$ is point from the previous subset, $(x^{'},y^{'})$ is point from the next subset, and $P$ is projection matrix for the first camera, and $P^{'}$ is projection matrix for the second camera which we compute in the previous part. 
	
	Then, we set $$A = \begin{pmatrix}
	yp_{3}^{T} - p_{2}^{T} \\
	p_{1}^{T} - xp_{3}^{T} \\
	y^{'}p_{3}^{'T} - p_{2}^{'T} \\
	p_{1}^{'T} - x^{'}p_{3}^{'T}
	\end{pmatrix}$$
	Then, use SVD to decompose $A$ and we could get $V^{T}$. And the location in 3D place should be the last column of $V$, which should be $X = (x,y,z,1)$ , and for each previous key points we get a $z$, for each next key points we get a $z^{'}$. For each candidate camera pose, we count the number when $z$ and $z^{'}$ are both positive.  Then the right relative should be the one with largest number. 
	

\section{Result}

\subsection{Essential Matrix}
Firstly, we compute the fundamental matrix, and we check the fundamental matrix with opencv's function 'findFundamentalMat'. We set the normalized matrix to be $\begin{pmatrix}
\frac{2}{h} & 0 & -1 \\
0 & \frac{2}{w} & -1 \\
0 & 0 & 1
\end{pmatrix}$

And we draw the epipolar constraint as below:

\begin{figure}[h]
	\centering
	\includegraphics[width=3.5in]{wrong.jpg}
	\caption{Nornalized-01}
\end{figure}

It seems their not quite the same. Then we change the normalization matrix into:$\begin{pmatrix}
\frac{3}{h} & 0 & -1.1 \\
0 & \frac{3}{w} & -1.1 \\
0 & 0 & 1
\end{pmatrix}$

Then we get:
\begin{figure}[h]
	\centering
	\includegraphics[width=3.5in]{correct.jpg}
	\caption{Nornalized-02}
\end{figure}

\begin{figure}[h]
	\centering
	\includegraphics[width=3.5in]{example.jpg}
	\caption{another example}
\end{figure}

I think the opencv's function doesn't use the same normalization as ours. And when we use different normalized matrix, the epipolar constraint becomes different. 

And then we check the fundamental matrix, after we correct the normalization we get:

Ours = 

$\begin{pmatrix}
-3.659150193881673e-07 & 0.0005045466575220245& 0.001278904965744987\\
-0.0005111755357057115 & 4.214772098631035e-08& 0.2299619944249809\\
-0.003350739034679 & -0.2297454482774965 &0.9999999999999999
\end{pmatrix}$ 

Opencv's = 

$\begin{pmatrix}
-3.770218424900359e-07& 0.0004595618639777839& -0.004903897467006754\\
-0.0004660854871973241&-3.327412881753011e-07& 0.2375100374095139\\
0.003193487642428392& -0.2369338437376147& 1
\end{pmatrix}$ 

They look quite similar to each other.

And essential should be $E = K^{T}FK = \begin{pmatrix}
1.01874798e+05 & 1.19131229e+05 &-7.34677901e+01\\
4.38130472e+04 & 6.51566884e+04& -5.86572486e+01\\
7.35402201e+01 & 5.87124923e+01 & 4.21477210e-08\\
\end{pmatrix}$

\subsection{Relative Pose}
The relative pose should be:
Rotation matrix = $\begin{pmatrix}
0.00315& -0.00349& 0.61596\\
0.00408& -0.00474& 0.049428 \\
0.00819&-0.00568& 0.99995
\end{pmatrix}$ 
Transition matrix = $[0.51324; 0.41504; 0.83249]$

We check the correctness with 'test five points'. 
And the rotation matrix we get is:

$\begin{pmatrix}
0.9999561864028342& -0.001081652661562135& -0.009298134340809422\\
0.001019461911297808& 0.9999770976290041& -0.006690658778073656 \\
0.009305158360362123&0.006680886542438612& 0.9999343877389635
\end{pmatrix}$ 

The correct one should be $\begin{pmatrix}
1& 0& 0\\
0& 1& 0\\
0&0& 1
\end{pmatrix}$

And the transition matrix we get is 

$[0.04262558990908046;
0.03707821835812766;
0.9984028569712178]$. 

And the correct one should be $[0;0;1]$

They are almost the same.

\subsection{Triangulated 3d map points}
Suppose the triangulated 3d map points is $X$, then $x = PX$, $X = P^{-1}x$. $P$ should be the correct relative pose between two frames.
%And we set some examples here.

%In the previous image, $x = $, and its corresponding 3D location should be $X = $.

%In the next image, $x = $, and its corresponding 3D location should be $X = $.


	
\end{document}